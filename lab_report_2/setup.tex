\documentclass[main.tex]{subfiles}
\begin{document}

\section{Experimental setup}

% Quantum efficiency of \SI{50}{\percent}.
% Passband filer at \SI{810}{nm}. 
% Jitter at \SI{1}{ns}. 
% Dark count at a few times \SI{100}{Hz}. 

We give a short description of the experimental apparatus, neglecting all mirrors, lenses, irises and in general devices needed in order to direct, select and concentrate the light beam.

A blue ``pump'' laser emits light with wavelength \(\lambda = \SI{405}{nm}\) continuously.
The light from this laser is initially horizontally polarized; it passes through a half-wave plate which achieves a linear polarization at \SI{45}{\degree} from the horizontal: a state proportional to \(\ket{H} + \ket{V}\), where \(H\) and \(V\) denote the horizontal and vertical linear polarizations respectively. 

The light then impinges upon a pair of parametric fluorescence crystals with optical axes orthogonal to each other and to the propagation direction;
a fraction \(\num{e-8}\divisionsymbol \num{e-7} \) of the times an incoming photon from the pump is turned into a pair of photons with \(\lambda = \SI{810}{nm}\) and with propagation directions of around \SI{4}{\degree} from the original beam. 

The polarizations of these two photons are entangled: their state is 
%
\begin{align} \label{eq:epr-state}
\frac{1}{\sqrt{2}} \qty(\ket{HH} + \ket{VV})
\,,
\end{align}
%
where \(\ket{H}\) or \(\ket{V}\) denotes the horizontal or vertical polarization of a photon. 

These two photons then each pass through a half-wave plate and a Polarizing Beam Splitter: this pair of components selects a (linear) polarization to project the state along, since the half-wave plate rotates the polarization by a certain angle, while the PBS transmits only photons with a certain polarization (say, \(H\)).
The reflected photons are discarded in this setup. 

In order to measure along the polarization \(\cos \theta \ket{H} + \sin \theta \ket{V}\) we must rotate the polarization of the photons by an angle \(\theta \), which is achieved by rotating the half-wave plate by an angle \(\theta /2\). 

The half-wave plate at an angle \(\varphi \) can be modelled as the quantum gate 
%
\begin{align}
\left[\begin{array}{cc}
\cos(2 \varphi ) & \sin(2 \varphi ) \\ 
- \sin(2 \varphi ) & \cos(2 \varphi )
\end{array}\right]
\end{align}
%
in the single-photon polarization space with basis \(\qty{\ket{H}, \ket{V}}\).

After the PBS, the photons reach a single-photon detector, which measures their arrival time with a jitter of approximately \SI{1}{ns}, a quantum efficiency (probability that a specific photon is detected) of approximately \SI{50}{\percent}, and a dark count of approximately \SI{100}{Hz}. 

\end{document}