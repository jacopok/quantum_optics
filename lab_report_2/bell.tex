\documentclass[main.tex]{subfiles}
\begin{document}

\section{Bell test}

% Bell, or CHSH inequalities are 
A Bell test allows us to experimentally violate a classical inequality, showing that a theory with classical probability (and possibly hidden variables) cannot describe a quantum system --- or, at least, entanglement in the polarizations of photons. 

The two photons can be measured in two different bases (denoted as \(x=0\) and \(x=1\) for the first photon, \(y=0\) and \(y=1\) for the second) and can yield two different outcomes, which we associate to the values \(+1\) and \(-1\) when computing expectation values.\footnote{We denote these, instead, as \(0\) and \(1\) when writing results in matrix notation, for consistency with the indexing of vectors.} We can then define 
%
\begin{align}
S = \expval{00} + \expval{01}+ \expval{10} - \expval{11}
\,,
\end{align}
%
where by \(\expval{ij}\) we mean \(\expval{x=i, y=j}\). 

Classically, it can be shown \cite[]{clauserProposedExperimentTest1969} that \(\abs{S} \leq 2\), while a quantum system can theoretically achieve \(\abs{S} = 2 \sqrt{2} \approx \num{2.82}\). 

The two bases\footnote{They are not bases for the full two-photon Hilbert space, their span only consists of the linear polarizations, which is enough for our purposes. } are chosen for the first photon to be the ones corresponding to the maximum quantum violation of the inequalities: the first is \(\qty{\ket{H}, \ket{V}}\), while the second is \(\qty{\ket{D}, \ket{A}}\), where the states \(\ket{A}\) and \(\ket{D}\) are defined as \(\qty(\ket{H} \pm \ket{V}) / \sqrt{2}\).
For the second photon the bases are the same but rotated around the propagation direction by an angle \(\pi /8\).

Measurements are performed by selecting one of these four vectors for each of the photons (so, performing \(4^2\) measurements for all the combinations) through a half-wave plate and counting the number of photons being detected after it per unit time. 

The detection numbers are turned into expectation values by estimating probabilities through their corresponding relative frequencies, and then computing \(\expval{x} = \sum_i x_i \mathbb{P}(x_i)\). 

Experimentally, with this method we achieved \(\abs{S} = \num{2.597(15)}\).

An explanation of how the data was analyzed, going from the list of the arrival times to the Bell inequality violation, can be found in a Jupyter notebook at \url{https://nbviewer.jupyter.org/github/jacopok/quantum_optics/blob/master/bell_test/report_bell_test.ipynb}.

The code for the analysis can be found in the folder \url{https://github.com/jacopok/quantum_optics/tree/master/bell_test}. 

\end{document}
