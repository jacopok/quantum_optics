\documentclass[main.tex]{subfiles}
\begin{document}

\section{Quantum Key Distribution}

With the same experimental setup as the Bell test, we were able to implement a Quantum Key Distribution entanglement-based scheme, first described by \textcite{bennettQuantumCryptographyBell1992}. 

The two conventional communicators Alice and Bob can achieve a shared secret key by receiving entangled qubits. 
Two bases are agreed upon, we use the same ones which were used in the Bell inequalities case (\(H-V\) and \(D-A\)).
They measure each qubit in a randomly-chosen basis, half of the time on average they will happen to choose the same one.
They can later disclose which bases were used; the results of the measurements in a shared basis then should be a shared message. 

In order to be sure that the qubits have not been tampered with a certain fraction of them should be periodically analyzed publicly: if they manage to retain entanglement (which can be verified, for example, by checking for the violation of Bell inequalities) then they are safe. 

We measured the rate of key transfer which can be achieved by our experimental setup. 
The state we use is the same as in \eqref{eq:epr-state}, and it can also be written as 
%
\begin{align}
\frac{1}{\sqrt{2}} \qty(\ket{HH} + \ket{VV})
=
\frac{1}{\sqrt{2}} \qty(\ket{AA} + \ket{DD})
\,.
\end{align}

The key rate is constrained by errors, which can be quantified through the Quantum Bit Error Rate: the probability of detecting \(HV\), \(AD\) and such should be zero ideally, but because of various sources of noise this is not so.
Then, for the \(H-V\) basis we define 
%
\begin{align}
\text{QBER}(HV) = \mathbb{P}(HV)+ \mathbb{P}(VH)
\,,
\end{align}
%
and similarly for \(A-D\). 
The fraction of usable qubits is then calculated as 
%
\begin{align}
r = 
1
- h (\text{QBER}(HV))
- h (\text{QBER}(AD))
\,,
\end{align}
%
where the function \(h\) comes from considerations about mutual information of Alice and Bob; it is given by 
%
\begin{align}
h (p) = - \qty(p \log_2 p + (1-p) \log_2 (1-p) )
\,.
\end{align}

With our experiment we find \(r \approx \num{0.72+-0.01}\). 

The rate in bits per second is further halved because of the random choice of basis of Alice and Bob.
With this consideration, we find a key rate of approximately \SI{125}{Hz} (or bits per second); this will be further reduced because of the requirement to reserve a fraction of the qubits for entanglement verification.

A Jupyter notebook summarizing the results and analysis can be found at \url{https://github.com/jacopok/quantum_optics/blob/master/QKD/report_QKD.ipynb}.

The full code for the analysis is in the folder \url{https://github.com/jacopok/quantum_optics/tree/master/QKD}.

Considering the relative simplicity of the experimental setup --- it is a table-top experiment, after all --- this key rate is not bad, it would be enough for somewhat slow text-based communication, at about 10 letters a second. 

\end{document}