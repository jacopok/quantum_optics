\documentclass[main.tex]{subfiles}
\begin{document}

\section{Indivisibility of the photon}

\subsection{Description of the apparatus}

\begin{figure}[ht]
\centering


\tikzset{every picture/.style={line width=0.75pt}} %set default line width to 0.75pt        

\begin{tikzpicture}[x=0.75pt,y=0.75pt,yscale=-1,xscale=1]
%uncomment if require: \path (0,300); %set diagram left start at 0, and has height of 300

%Shape: Rectangle [id:dp3902522124544314] 
\draw   (37,72) -- (111.5,72) -- (111.5,112) -- (37,112) -- cycle ;
%Shape: Wave [id:dp29533096809256154] 
\draw  [dash pattern={on 4.5pt off 4.5pt}] (112,93) .. controls (116.08,95.56) and (119.98,98) .. (124.5,98) .. controls (129.02,98) and (132.92,95.56) .. (137,93) .. controls (141.08,90.44) and (144.98,88) .. (149.5,88) .. controls (154.02,88) and (157.92,90.44) .. (162,93) .. controls (166.08,95.56) and (169.98,98) .. (174.5,98) .. controls (179.02,98) and (182.92,95.56) .. (187,93) .. controls (191.08,90.44) and (194.98,88) .. (199.5,88) .. controls (204.02,88) and (207.92,90.44) .. (212,93) .. controls (216.08,95.56) and (219.98,98) .. (224.5,98) .. controls (229.02,98) and (232.92,95.56) .. (237,93) .. controls (237.17,92.9) and (237.33,92.79) .. (237.5,92.69) ;
%Shape: Rectangle [id:dp5615185733257464] 
\draw   (238,73) -- (308,73) -- (308,113) -- (238,113) -- cycle ;
%Shape: Chord [id:dp5426551407812648] 
\draw   (307.1,184.93) .. controls (308.92,188.52) and (309.95,192.47) .. (310,196.63) .. controls (310.17,213.2) and (294.64,226.8) .. (275.31,227) .. controls (255.99,227.2) and (240.18,213.93) .. (240,197.37) .. controls (239.96,193.2) and (240.91,189.23) .. (242.65,185.61) -- cycle ;
%Shape: Wave [id:dp5480038928028264] 
\draw  [dash pattern={on 4.5pt off 4.5pt}] (309,91) .. controls (313.08,93.56) and (316.98,96) .. (321.5,96) .. controls (326.02,96) and (329.92,93.56) .. (334,91) .. controls (338.08,88.44) and (341.98,86) .. (346.5,86) .. controls (351.02,86) and (354.92,88.44) .. (359,91) .. controls (363.08,93.56) and (366.98,96) .. (371.5,96) .. controls (376.02,96) and (379.92,93.56) .. (384,91) .. controls (388.08,88.44) and (391.98,86) .. (396.5,86) .. controls (401.02,86) and (404.92,88.44) .. (409,91) .. controls (413.08,93.56) and (416.98,96) .. (421.5,96) .. controls (426.02,96) and (429.92,93.56) .. (434,91) .. controls (434.17,90.9) and (434.33,90.79) .. (434.5,90.69) ;
%Shape: Square [id:dp4933988712132191] 
\draw   (435,69) -- (485,69) -- (485,119) -- (435,119) -- cycle ;
%Straight Lines [id:da6356953181395324] 
\draw    (435,69) -- (485,119) ;
%Shape: Wave [id:dp43746939652005445] 
\draw  [dash pattern={on 4.5pt off 4.5pt}] (274.13,112.25) .. controls (271.88,116.33) and (269.75,120.23) .. (269.75,124.75) .. controls (269.75,129.27) and (271.88,133.17) .. (274.13,137.25) .. controls (276.37,141.33) and (278.5,145.23) .. (278.5,149.75) .. controls (278.5,154.27) and (276.37,158.17) .. (274.13,162.25) .. controls (271.88,166.33) and (269.75,170.23) .. (269.75,174.75) .. controls (269.75,178.44) and (271.17,181.71) .. (272.9,185) ;
%Shape: Wave [id:dp8229982636223081] 
\draw  [dash pattern={on 4.5pt off 4.5pt}] (460.13,120.25) .. controls (457.88,124.33) and (455.75,128.23) .. (455.75,132.75) .. controls (455.75,137.27) and (457.88,141.17) .. (460.13,145.25) .. controls (462.37,149.33) and (464.5,153.23) .. (464.5,157.75) .. controls (464.5,162.27) and (462.37,166.17) .. (460.13,170.25) .. controls (457.88,174.33) and (455.75,178.23) .. (455.75,182.75) .. controls (455.75,186.44) and (457.17,189.71) .. (458.9,193) ;
%Shape: Chord [id:dp06188698806334458] 
\draw   (494.22,193.27) .. controls (496.01,196.87) and (497,200.84) .. (497,205) .. controls (497,221.57) and (481.33,235) .. (462,235) .. controls (442.67,235) and (427,221.57) .. (427,205) .. controls (427,200.84) and (427.99,196.87) .. (429.78,193.27) -- cycle ;
%Shape: Wave [id:dp9566458935440425] 
\draw  [dash pattern={on 4.5pt off 4.5pt}] (484.75,91.63) .. controls (488.83,93.87) and (492.73,96) .. (497.25,96) .. controls (501.77,96) and (505.67,93.87) .. (509.75,91.63) .. controls (513.83,89.38) and (517.73,87.25) .. (522.25,87.25) .. controls (526.77,87.25) and (530.67,89.38) .. (534.75,91.63) .. controls (538.83,93.87) and (542.73,96) .. (547.25,96) .. controls (550.94,96) and (554.21,94.58) .. (557.5,92.85) ;
%Shape: Chord [id:dp22990401692182982] 
\draw   (557.27,60.78) .. controls (560.87,58.99) and (564.84,58) .. (569,58) .. controls (585.57,58) and (599,73.67) .. (599,93) .. controls (599,112.33) and (585.57,128) .. (569,128) .. controls (564.84,128) and (560.87,127.01) .. (557.27,125.22) -- cycle ;

% Text Node
\draw (48,84) node [anchor=north west][inner sep=0.75pt]   [align=left] {LASER};
% Text Node
\draw (253,85) node [anchor=north west][inner sep=0.75pt]   [align=left] {SPDC};
% Text Node
\draw (256,193) node [anchor=north west][inner sep=0.75pt]   [align=left] {Gate};
% Text Node
\draw (452,204) node [anchor=north west][inner sep=0.75pt]   [align=left] {R};
% Text Node
\draw (569,83) node [anchor=north west][inner sep=0.75pt]   [align=left] {T\\};
% Text Node
\draw (450,32) node [anchor=north west][inner sep=0.75pt]   [align=left] {BS};


\end{tikzpicture}
\caption{SPDC setup.}
\label{fig:spdc}
\end{figure}

\subsection{Data analysis}

The output of the timetagger is a table of times and corresponding channel values.
Each entry of this table represents a photon detection at a single detector.
The times are expressed as integer multiples of the temporal resolution of the timetagger, which is nominally \SI{80.955}{ps}.

There is a slight complication: the timetagger exhibits a slight preference for odd values of the integer which represents the time, as opposed to even ones. This is discussed in more detail in appendix [ADD REF], for now we can just say that this effect does not pose an issue for our analysis.

We can then 

\end{document}