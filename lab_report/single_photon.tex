\documentclass[main.tex]{subfiles}
\begin{document}

\section{Indivisibility of the photon}

\subsection{Description of the apparatus}

\begin{figure}[ht]
\centering
\input{figures/spdc.tex}
\caption{SPDC setup.}
\label{fig:spdc}
\end{figure}

\subsection{Data analysis}

The output of the timetagger is a table of times and corresponding channel values.
Each entry of this table represents a photon detection at a single detector.
The times are expressed as integer multiples of the temporal resolution of the timetagger, which is nominally \SI{80.955}{ps}.

There is a slight complication: the timetagger exhibits a slight preference for odd values of the integer which represents the time, as opposed to even ones. This is discussed in more detail in appendix [ADD REF], for now we can just say that this effect does not pose an issue for our analysis.

We can then 

\end{document}